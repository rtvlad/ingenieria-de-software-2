\documentclass[10pt,a4paper]{article}
\usepackage[latin1]{inputenc}
\usepackage[spanish]{babel}
\usepackage{multicol}
\usepackage{amsmath}
\usepackage{amsfonts}
\usepackage{amssymb}
\usepackage{enumerate}
\usepackage{eurosym}
\usepackage{graphicx}
\usepackage{anysize}
\usepackage{colortbl}
\pagestyle{empty}
\marginsize{3cm}{2cm}{2cm}{2cm}
\author{Jesus S�nchez-Oro Calvo}
\title{Complejos}
\begin{document}

\section{Definici�n del sistema}
\subsection{Determinaci�n del alcance del sistema}
Seg�n el Pliego de Condiciones T�cnicas y el an�lisis del sistema, se han identificado los siguientes casos de uso:
	\begin{enumerate}
	\item \textbf{Env�o de documentos:} En este caso de uso se van a incluir todas las posibles acciones relacionadas con el env�o de documentos por parte de un cliente, por lo que este caso de uso se subdivide en tres casos diferentes. Las diferencias entre los tres casos de uso est�n s�lo en las acciones internas realizadas para cada tipo de documento.
		\begin{enumerate}
		\item \textit{Env�o de cartas:} Este caso de uso representa la funcionalidad del env�o de cartas a trav�s del portal. Comprende la acci�n desde que el usuario inicia un nuevo env�o hasta que recibe la confirmaci�n del mismo.
		\item \textit{Env�o de telegramas:} Este caso de uso representa la funcionalidad del env�o de telegramas a trav�s del portal. Comprende la acci�n desde que el usuario inicia un nuevo env�o hasta que recibe la confirmaci�n del mismo.
		\item \textit{Env�o de fax:} Este caso de uso representa la funcionalidad del env�o de fax a trav�s del portal. Comprende la acci�n desde que el usuario inicia un nuevo env�o hasta que recibe la confirmaci�n del mismo.
		\end{enumerate}
	
	
	\end{enumerate}

\end{document}