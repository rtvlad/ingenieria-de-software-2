\documentclass[10pt,a4paper]{article}
\usepackage[latin1]{inputenc}
\usepackage[spanish]{babel}
\usepackage{multicol}
\usepackage{amsmath}
\usepackage{amsfonts}
\usepackage{amssymb}
\usepackage{enumerate}
\usepackage{eurosym}
\usepackage{graphicx}
\usepackage{anysize}
\pagestyle{empty}
\marginsize{3cm}{2cm}{2cm}{2cm}
\author{Jesus S�nchez-Oro Calvo}
\title{SRS}
\begin{document}

\section{Introducci�n}

\subsection{Prop�sito}
	El proyecto tiene como prop�sito el desarrollo de un nuevo portal de Correos y de un portal filat�lico. Este documento est� dirigido a los responsables t�cnicos de Correos, para el consenso de los requisitos entre la empresa solicitante y Correos.
	
\subsection{Alcance}
	El portal web se denominar� Correos. Este portal permitir� a los usuarios realizar las operaciones relacionadas con la gesti�n postal que podr�a realizar en una oficina f�sica. Adem�s, permite la localizaci�n de env�o y dispone de un sistema de atenci�n al cliente encargado de solucionar aquellos problemas que se le pudieran presentar.
	
	La empresa de Correos se beneficiar� de dicho portal agilizando la mayor�a de los tr�mites que no requieran de un empleado para realizarse, permitiendo al personal dedicarse a aquellas transacciones que necesiten de una persona para su correcto funcionamiento. El cliente podr� gracias al portal realizar aquellas transacciones que lo permitan de manera autom�tica y sin necesidad de acudir a la oficina.
	
	El objetivo principal es el desarrollo del portal con un funcionamiento estable y capaz de atender las peticiones de todos los clientes que accedan a �l.
	
\subsection{Definiciones, siglas y abreviaciones}

	PENDIENTE
	
\subsection{Referencias}
	\begin{itemize}
	
	\end{itemize}

\end{document}